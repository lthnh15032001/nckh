\documentclass{article}
\usepackage[utf8]{vietnam}
\usepackage{graphicx}
\begin{document}

\section{DNS security}
\subsection{DNS Security là gì?}
\textbf{DNS Security (bảo mật DNS)}
là phương pháp giúp xác định các IP xấu, độc hại trên internet mà server truy cập. Giúp máy chủ tránh trở thành mục tiêu tấn công của kẻ xấu, đánh cắp thông tin và gây thiệt hại cho toàn bộ hệ thống của doanh nghiệp

\subsection{Tại sao DNS lại quan trọng}
Standard DNS query được sử dụng cho hầu hết các lưu lượng truy cập vào trang web. Do đó, chúng tạo cơ hội cho việc khai thác, chiếm quyền điều khiển DNS và tấn công trên on-path. Những cuộc tấn công này có thể chuyển hướng đến một bản sao giả mạo của trang web. Điều này sẽ thu thập thông tin của người dùng và khiến các doanh nghiệp phải chịu trách nhiệm. Và DNS Security có vai trò ngăn chặn những việc đó. Một trong những cách được biết đến để bảo vệ khỏi các mối đe dọa DNS là áp dụng giao thức DNSSEC.

\subsection{DNSSEC là gì?}

\begin{figure}[!htp]
    \centering
    \includegraphics[scale=0.5]{Image/a.png}
    \caption{Caption}
    \label{fig:my_label}
\end{figure}
Giống như nhiều protocol Internet khác, hệ thống DNS không được thiết kế với khả năng bảo mật và sẽ có một số hạn chế về thiết kế. Điều này khiến các hacker dễ dàng chiếm đoạt DNS lookup cho các mục đích xấu. Chẳng hạn như đưa người dùng đến một trang web lừa đảo có thể phát tán các phần mềm malware độc hại. Hoặc thậm chí thu thập thông tin các nhân.

DNS Security Extensions (DNSSEC) là một giao thức bảo mật được tạo ra để giảm thiểu vấn đề này. DNSSEC bảo vệ, chống lại các cuộc tấn công bằng cách sử dụng chữ ký điện tử vào các dữ liệu, giúp bảo vệ tính hợp lệ của nó. Để đảm bảo việc tìm kiếm an toàn, việc ký phải diễn ra ở mọi cấp trong quy trình DNS lookup. Từ đó đảm bảo data không bị giả mạo.

DNSSEC triển khai chính sách chữ ký số phân cấp trên tất cả các layer DNS. Mặc dù sự cải thiện về bảo mật là ưu tiên. Nhưng DNSSEC được thiết kế để tương thích ngược, đảm bảo DNS lookup giải quyết các vấn đề chính xác. DNSSEC có nhiệm vụ hoạt động với biện pháp bảo mật khác như SSL/TLS. Đây là một phần của chiến lược bảo mật Internet toàn diện.

DNSSEC tạo ra một hệ thống tin cậy parent-child đến tận DNS root zone. Chuỗi tin cậy này không thể bị xâm nhập ở bất kỳ lớp DNS nào.

Để đóng chuỗi tin cậy, DNS root zone cần được xác thực. Thật thú vị, cái được gọi là Root Zone Signing Ceremony sẽ chọn các cá nhân trên khắp thế giới để ký DNSKEY RRset root một cách công khai và được kiểm tra rõ ràng.

\subsection{Một số cuộc tấn công phổ biến liên quan đến DNS}

DNSSEC là một giao thức có tính bảo mật vô cùng mạnh. Nhưng tiếc là nó chưa được áp dụng rộng rãi. Trên thực tế DNS là một phần không thể thiếu trong hầu hết các Internet request. Điều này khiến DNS trở thành mục tiêu chính cho các cuộc tấn công. Những hacker đã tìm ra một số cách để nhắm mục tiêu và khai thác DNS server. Dưới đây là một số cách phổ biến nhất:


\textbf{DNS spoofing/cache poisoning}

Trong các cuộc tấn công này, DNS data sẽ được đưa vào cache DNS. Điều này dẫn đến việc trình phân giải trả về địa chỉ IP không chính xác cho một domain. Thay vì truy cập đúng trang web, lưu lượng truy cập sẽ được chuyển hướng đến một trang web độc hại hoặc bất kỳ nơi nào mà hacker muốn. Thường đây sẽ là bản sao của trang web gốc, sử dụng cho các mục đích xấu. Chẳng hạn như phát tán phần mềm độc hại hoặc thu thập thông tin cá nhân.

\textbf{DNS tunneling}

Cuộc tấn công này sử dụng các giao thức khác và sử dụng kỹ thuật tunneling thông qua DNS query và những phản hồi của DNS. Các hacker có thể sử dụng SSH, TCP hoặc HTTP để chuyển phần mềm độc hại hoặc thông tin bị đánh cắp vào DNS query. Điều này sẽ không bị phát hiện bởi hầu hết các firewall.

\textbf{DNS hijacking}

Các hacker sẽ chuyển hướng các DNS query đến một server với tên miền khác. Điều này có thể được thực hiện bằng phần mềm độc hại hoặc sửa đổi DNS server. Mặc dù kết quả là tương tự như DNS spoofing, nhưng cách thức tấn công thì khác. Cụ thể, nó sẽ nhắm vào các DNS record của trang web trên nameserver.

\textbf{Tấn công NXDOMAIN}

Đây là một kiểu tấn công DNS flood. Trong đó, các hacker sẽ flood DNS server bằng cách liên tục request những record không tồn tại. Điều này nhằm gây ra lỗi từ chối dịch vụ DDoS cho lưu lượng truy cập hợp pháp. Việc này có thể được thực hiện bằng cách sử dụng công cụ tấn công tinh vi. Chúng có thể tự động tạo subdomain cho mỗi request. Các cuộc tấn công NXDOMAIN cũng có thể nhắm vào các trình phân giải đệ quy. Mục đích là nhằm lấp đầy bộ nhớ cache của trình phân giải với các request rác.

\textbf{Tấn công Phantom Domain}

Một cuộc tấn công Phantom Domain sẽ cho kết quả tương tự như tấn công NXDOMAIN trên trình phân giải DNS. Kẻ tấn công thiết lập nhiều server Phantom domain có thể phản hồi các request rất chậm hoặc không phản hồi. Sau đó, trình phân giải bị ảnh hưởng bởi một loạt request đến các domain này và buộc phải cờ phản hồi. Điều này dẫn đến hiệu suất chậm và bị từ chối dịch vụ.

\textbf{Tấn công một subdomain ngẫu nhiên}

Trong trường hợp này, kẻ tấn công gửi DNS query cho một số subdomain ngẫu nhiên, không tồn tại của một trang web hợp pháp. Mục đích là tạo ra DDoS cho nameserver có thẩm quyền của domain. Khi đó, bạn không thể tra cứu trang web từ nameserver. ISP phục vụ kẻ tấn công cũng có thể bị ảnh hưởng. Vì bộ nhớ cache của trình phân giải đệ quy của chúng sẽ bị tràn ngập với các bad request.

\textbf{Tấn công Domain lock-up}

Những kẻ tấn công sẽ thiết lập các domain và trình phân giải đặc biệt để tạo kết nối TCP. Khi các trình phân giải mục tiêu gửi request, các domain này sẽ gửi lại các packet ngẫu nhiên. Do đó sẽ khóa các resource của trình phân giải.

\textbf{Tấn công CPE dựa trên botnet}

Các cuộc tấn công này được thực hiện bằng cách sử dụng CPE. Những kẻ tấn công xâm phạm CPE và thiết bị trở thành một phần của mạng botnet. Các botnet này được sử dụng để tấn công subdomain hoặc domain ngẫu nhiên.

\section{Exploit attack}
\subsection{Exploit là gì?}

Exploit là một loại chương trình, được tạo ra để nhắm vào một điểm yếu nhất định nào đó – gọi là lỗ hổng – trong một phần mềm hoặc phần cứng. Nói rộng ra thì định nghĩa của exploit bao gồm tất cả mọi thứ, từ ứng dụng phần mềm đến các chuỗi code, dữ liệu, hay thậm chí là chỉ các lệnh đơn giản.

Nói một cách khác, exploit là một cách cho phép hacker tận dụng lỗ hổng bảo mật để đạt được lợi ích nhất định nào đó. Nếu một người có thể lập trình nó, và dùng nó để tận dụng lỗ hổng bảo mật của phần cứng hay phần mềm, thì đó là chính là exploit.

\subsection{Exploit hoạt động như thế nào?}
Vậy cách thức hoạt động của exploit là gì? Trước hết, việc khai thác phần mềm không thể được thực hiện nếu không có lỗi thiết kế ở trong phần mềm đó. Một khi các hacker đã xác định được lỗ hổng này, họ có thể viết một chương trình để thực hiện tấn công exploit vào nó.

Có nhiều hacker sử dụng exploit để phát tán malware. Sau đây là một tình huống mà ta có thể dễ dàng bị tấn công exploit và nhiễm malware:


$\bullet$ Giả sử bạn vô tình truy cập vào một trang web có quảng cáo độc hại trong quá trình truy cập internet. Quảng cáo này bề ngoài trông có vẻ vô hại, nhưng thực chất nó lại có một exploit kit đang scan máy của bạn để tìm ra những điểm yếu.

$\bullet$ Sau khi tìm thấy bất kỳ lỗ hổng nào, quảng cáo đó sẽ thực hiện khai thác để truy cập vào máy tính thông qua các lỗ hổng đó. Tiếp đến, chúng sẽ bắt đầu phát tán malware vào trong hệ thống. Khi exploit được dùng cho mục đích cài đặt malware, thì malware đó sẽ được gọi là payload.

Xét về cấp độ kỹ thuật, cyber exploit (khai thác không gian mạng) lại không được xem là một malware, sở dĩ vì chúng chẳng có gì độc hại cả. Sự nguy hiểm của việc khai thác đến từ chính những hành động của hacker sau khi malware xâm nhập vào hệ thống. Không có một thuật ngữ nhất định về “exploit virus”, như ransomware hay virus. Tuy nhiên, tấn công exploit chủ yếu được dùng để phát tán malware trong các cuộc tấn công lâu dài.

\subsection{Các kiểu Exploit phổ biến}

Có càng nhiều lỗ hổng bảo mật thì sẽ có càng nhiều khả năng khai thác phần mềm. Do đó, hầu như mỗi ngày ta đều phát hiện ra các cuộc tấn công mới. Vậy các loại exploit là gì? Exploit có thể được chia ra thành hai loại, phụ thuộc vào việc lỗ hổng bảo mật đã được khắc phục hay chưa.

\textbf{Known exploit}

Khi ai đó phát hiện ra lỗ hổng bảo mật của phần mềm, họ thường sẽ thông báo cho developer của phần mềm đó để có thể nhanh chóng đưa ra bản vá khắc phục lỗ hổng đó. Hoặc họ cũng có thể lan truyền về lỗ hổng đó trên internet để cảnh báo những người dùng khác. Dù bằng cách nào đi chăng nữa, các developer đều được mong chờ có thể phản hồi và nhanh chóng khắc phục lỗ hổng đó, trước khi có bất kỳ hacker nào tìm ra cách khai thác.

Tiếp đến, các bản vá (patch) này sẽ được đưa cho người dùng thông qua các bản cập nhật phần mềm. Đây cũng chính là lý do vì sao ta cần phải liên tục cập nhật phần mềm của mình. Bất kỳ cách khai thác nào nhắm vào lỗ hổng đã được vá đều được gọi là known exploit (khai thác đã biết). Vì mọi người đều đã biết được về lỗ hổng bảo mật đó.

\textbf{Zero-day exploit (Unknown exploit)}

Đôi khi, exploit có thể đến rất bất ngờ. Khi một hacker phát hiện ra lỗ hổng bảo mật nào đó, và ngay lập tức tìm ra cách để khai thác nó, thì được gọi là zero-day exploit (khai thác chưa biết). Bởi vì exploit diễn ra rất nhanh sau khi tìm ra được lỗ hổng bảo mật.

Zero-day exploit rất nguy hiểm, vì không hề có giải pháp rõ ràng hay tức thời nào cho các lỗ hổng bảo mật. Chỉ có những kể tấn công mới phát hiện ra các lỗ hổng đó, và cũng chỉ họ mới biết cách để khai thác nó. Để có thể đối phó với kiểu tấn công này, các developer bắt buộc phải đưa ra bản vá, tuy nhiên vẫn không thể bảo vệ những người đã bị nhắm mục tiêu trước đó.

\textbf{Hardware exploit (Khai thác phần cứng)}

Mặc dù chủ yếu chúng ta chỉ nghe đến khai thác phần mềm, điều này không có nghĩa rằng đó là loại khai thác duy nhất. Đôi khi, các hacker thậm chí còn có thể khai thác các lỗ hổng bảo mật ở trong phần cứng vật lý (và cả firmware) trong thiết bị.

Meltdown và Spectre là hai loại lỗ hổng phần cứng nổi tiếng nhất, bởi mức độ nghiêm trọng của chúng. Phạm vi của Meltdown chỉ bị giới hạn ở những thiết bị sử dụng processor của Intel. Mặt khác, lỗ hổng Spectre lại hiện diện ở mọi processor.

May mắn thay, hiện nay vẫn chưa có cách nào để khai thác hai loại lỗ hổng này. Đồng thời, Intel và những nhà sản xuất khác cũng đã nhanh chóng đưa ra các bản vá để giảm thiểu rủi ro.

\subsection{Những đối tượng nào dễ bị tấn công Exploit}
Vậy những mục tiêu “lý tưởng” của exploit là gì? Đó chính là những người dùng không bao giờ chịu cập nhật phần mềm. Những phần mềm có trên thì trường càng lâu thì các hacker càng có nhiều thời gian để tìm ra các lỗ hổng, cũng như cách để khai thác nó.

Lấy ví dụ như RIG, Magnitude, và Neutrino – đều dựa trên những phần mềm đã lỗi thời như Internet Explorer hay Adobe Flash. Sau đó, WannaCry và NotPetya đã tận dụng khai thác lỗ hổng bảo mật EternalBlue. Những người dùng chưa cập nhật phần mềm đều đã chịu những ảnh hưởng nghiêm trọng.

Đối với Zero-day exploit, thì đây lại là ngoại lệ. Bởi vì không hề có bất kỳ cảnh báo nào, cũng như không có cơ hội để cập nhật, vì vậy tất cả người dùng đều có khả năng trở thành nạn nhân.
\end{document}